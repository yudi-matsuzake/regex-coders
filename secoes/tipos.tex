\section{Tipos de firewall}

\begin{frame}
	\frametitle{Tipos de firewall}
	\begin{itemize}
		\item Primeira geração: Filtros de pacote (Filtros "\textit{Stateless}");

		\item Segunda geração: Filtros "\textit{Stateful}";

		\item Terceira geração: Camada de aplicação.
	\end{itemize}
\end{frame}


%----------------------------
%FUNCIONAMENTO
%QUAIS SÃO OS ATRIBUTOS DAS POLÍTICAS?
%Quais camadas?
%EXEMPLOS

\subsection{Filtros de pacote (stateless)}

\begin{frame}
	\frametitle{Funcionamento}
	\begin{itemize}
		\item Seu funcionamento é baseado na filtragem de pacotes; (ah, vá!)
		\item Seu funcionamento é simples:
		\begin{itemize}
			\item Se o pacote casar com as políticas do firewall:
				\begin{itemize}
					\item O pacote entra na rede na maior tranquilidade;
				\end{itemize}
			\item Se o pacote não casar com as políticas do firewall:
				\begin{itemize}
					\item O pacote é descartado;
				\end{itemize}
		\end{itemize}
	\end{itemize}
\end{frame}

\begin{frame}
	\frametitle{Políticas}
	\begin{itemize}
		\item Esse tipo de firewall só verifica a informação do pacote em si;
		\item i.e. o firewall não guarda a informação do status da conexão;
		\item Geralmente atributos verificados são:
			\begin{itemize}
				\item IP de origem e IP de destino
				\item O protocolo (UDP e TCP)
				\item A porta
			\end{itemize}
	\end{itemize}
\end{frame}

\begin{frame}
	\frametitle{Modelo OSI}

	\begin{itemize}
		\item O firewall stateless verifica informações das camadas física, enlace, rede e um pouquinho da camada de transporte;
	\end{itemize}

\end{frame}

\begin{frame}
	\frametitle{Exemplos}
	\begin{itemize}
		\item ACL - \textit{Access Control List}
		\item iptables \footnote{O iptables por si só não é um firewall stateful. Mas o iptables se torna uma ferramenta muito fácil de se \textit{scriptar} um firewall stateful};
	\end{itemize}
\end{frame}

\subsection{Filtros de estado de conexão (stateful)}

\begin{frame}
	\frametitle{Funcionamento}
	\begin{itemize}
		\item Um firewall stateful é qualquer firewall que consegue fazer uma Inspeção de Pacotes Stateful \footnote{Traduzido livremente de SPI - Stateful Packet Inspection};
		\item Firewalls Stateful trabalham fortemente na camada de transporte, pois guardam o estado das conexões TCP e UDP;
		\item Guardando os estados das conexões, esses firewalls conseguem gerar atributos significantes de cada conexão e gerenciá-las;
		\item Eles também tem um banco de dados de possíveis sessões de ataques, se essa sessão não é permitida o pacote é dropado;
		\item Sistema para não encher a memória de sessões apenas inicializadas apartir do three-way handshake;
		\item A maioria dos firewalls mais modernos são statefull;
	\end{itemize}
\end{frame}


\subsection{Camada de aplicaçao}

\begin{frame}
	\frametitle{Funcionamento}
	\begin{itemize}
		\item São chamados de Firewalls da Próxima Geração (NGFW\footnote{Next-generation Firewall});
		\item Trabalham na camada de aplicação e "entendem" como os protocolos dessa camada funcionam (e.g FTP, DNS, HTTP);
		\item Inspecionam profundamente os pacotes e seções DPI\footnote{Tradução livre de \textit{deep packet inspection}};
	\end{itemize}
\end{frame}

\begin{frame}
	\frametitle{Funcionalidades extendidas}
	As três funcionalidades que DEVEM ter em um NGFW por causa do DPI e não tem em um stateful firewall é:
	\begin{enumerate}[i]
		\item Sistema de Prevenção de Intrusão (IPS\footnote{Intrusion Prevention System});
		\item Integração a Identificação do Usuário (UII\footnote{User Identity Integration});
		\item Firewall de Aplicação Web (WAF \footnote{Web Application Firewall}).
	\end{enumerate}
\end{frame}
